%!TEX root = josecon_charter.tex

\subsection{Foundational principles}
%%%%%%%%%%%%%%%%%%%%%%%%%%%%%%%%%%%%%%%%%%%%%%%%%%%%%%%%%%%%%%%%%%%

JOSEcon is a developer-friendly, free and open-access, peer-reviewed journal for 
economic software packages.

JOSEcon accepts submissions of \emph{code} written in conjunction with economic
research (implementations of the original numerical methods, estimation algorithms, 
replications of the existing results and methods),
and publishes \emph{code papers} which are short description of the purpose of the submitted 
software along with an example of use.

Requirements for the code paper are minimal to minimize additional effort needed to
submit to JOSEcon.

The review process has two stages. First stage aims at assessing the contribution
of the submitted code in moving forward the open source economic code base.
This assessment is performed by one or several editors, possibly with additional 
anonymous referee or referees.

The second stage of the review process is public and focuses on code review. 
Before acceptance decision is made, the assigned referee or referees make sure that 
the submitted software satisfies the list of known JOSEcon requirements.

The accepted code papers are published on \href{http://josecon.org}{JOSEcon website}
along with a link to the public repository containing the accepted code itself.
The published code paper is assigned a digital object identifier (DOI) with CrossRef, 
which can be displayed in the repository with a JOSEcon badge.

JOSEcon accepts for publication only open source code as per 
the \href{https://opensource.org/osd}{OSI definition}.

There are no restrictions on the programming languages at JOSEcon: submissions of code 
written in any languages are welcome, and the popularity of the language does not 
affect the outcome of the submission (apart, perhaps, from how fast the reviewers can be
appointed).

There are no restrictions on the technology and hardware required for running
the submitted code (apart from how fast the reviewers can be appointed).

The software should be feature complete (no half-baked solutions) and designed for 
maintainable extension (not one-off modifications). 
Minor ‘utility’ packages, including ‘thin’ API clients, are not acceptable.

The software should be available through public repository (GitHub, GitLab, etc),
namely:
\begin{itemize}
\item Be stored in a repository that can be cloned without registration
\item Be stored in a repository that is browsable online without registration
\item Have an issue tracker that is readable without registration
\item Permit individuals to create issues/file tickets against your repository
\end{itemize}

JOSEcon fosters the development of a community of computational economists, who may be
involved with a journal to a varying degree (from developers to users of the economic code,
from editors to reviewers to simply campaigners), but who share similar views about the
future of computations in economics, and are responsive and welcoming to the new 
members of the community.

Authors, editors and reviewers, as well as any other members of JOSEcon community,
adhere to a code of conduct adapted from the 
\href{http://contributor-covenant.org/}{Contributor Covenant} code of conduct.

Authors, editors and reviewers of JOSEcon are expected and required to adhere to ethical
standards, in particular:
\begin{itemize}
\item All authors are required to make the JOSEcon editors and the community aware of any 
mistakes in their published work which they become aware.
\item Plagiarism for both code and code papers is not tolerated.
\item Repeated submissions of the same work is not allowed. 
\item Submissions of the work which is under revision in a separate software journal is not allowed. 
(It is, however, perfectly fine to submit the ``big'' paper to a traditional journal, while the software
developed for it being submitted to JOSEcon, perhaps with a smaller scale example of use.)
\item Listed authors on the code papers must include all people who contributed towards the submission, 
and not include anybody who has not. 
For the replication papers JOSEcon encourages but does not require the inclusion of the original authors.
\item Code reviews should be accurate and non-fraudulent. 
\item Reviewers and editors must disclose conflicts of interest.
\end{itemize}

JOSS is an open access journal. Copyright of JOSS papers is retained by submitting authors and accepted papers are subject to a Creative Commons Attribution 4.0 International License.

Any code snippets included in JOSS papers are subject to the MIT license regardless of the license of the submitted software package under review.

Any use of the JOSS logo is licensed CC BY 4.0.


\subsection{Types of JOSEcon papers}
%%%%%%%%%%%%%%%%%%%%%%%%%%%%%%%%%%%%%%%%%%%%%%%%%%%%%%%%%%%%%%%%%%%

JOSEcon publishes three types of code papers:

\begin{enumerate}

\item Full featured code library or package that implements a particular model or 
solution/estimation method used in economics.
It is expected that a package has a well documented API, and is accompanied by
an independent run script that implements an example (or multiple examples) of use
of the package.
The code paper describes the purpose of the software, and presents the computed 
example.
Generally there is no single traditional paper associated with this type of submission, 
but is it understood that the research applications of the software would follow and build
on the provided examples.

\item Code supplement for a particular traditional paper that relies heavily on computations.
The scenario under the second type of submission is that the code written during the work
on a bigger paper is of a sufficient quality to constitute a publication at JOSEcon.
The primary criterion here is whether the code can be used in a similar application, or
serve as the bases for development of the next generation software in the area.
It is expected that the code is organized into a re-usable library/package, and a run
script that calls the routines in the library to produce the results in the paper.
Ideally, the whole set of results should be reproduced by the run script, but 
in case where the required run time would hinder the review process, simpler examples
are sufficient.

\item Replications of the existing methods and published results, written in new
programming languages, or with significant refactoring and improvements in code architecture.
JOSEcon welcomes replications of existing papers, but has special rules requiring
the authors of the replicated paper be contacted and invited to participate in the 
replication.  It is not necessary that they do, but the record of communication should
be submitted together with the code paper.
The replication code should also be organized into library/package and a run script
structure, and it is expected that the run script replicates the results in the same
form as in the original paper for easier comparison.

\end{enumerate}

Many other types and kinds of software written by or for economists are not currently
considered for publication at JOSEcon. 
In particular, the journal does not accept:
\begin{itemize}
\item Simple technical code such as thin library wrappers, converters, or similar;
\item Text markup and style files such as latex or html;
\item Code written exclusively for teaching purposes; \note{Fedor: refereeing contribution of teaching materials is all different matter}
\end{itemize}

\emph{Jupyter notebooks} and similar document format that allows for combination of executable 
code cells with explanatory comments, is an excellent way to present the API of a library and set
up examples. 
JOSEcon encourages the use of this formal for the run scripts that can simultaneously present 
the example.\note{Fedor: static pdf version of Jupyter notebooks can be used for code paper?}


\subsection{Audience and role models}
%%%%%%%%%%%%%%%%%%%%%%%%%%%%%%%%%%%%%%%%%%%%%%%%%%%%%%%%%%%%%%%%%%%

JOSEcon offers several role models for a few target audiences.

The primary audience (JOSEcon community) are the economists who already rely on computational
methods or plan to use them in the near future: faculty and students writing code in their 
research projects.

Members of the JOSEcon community fill the roles on the editorial board and referees, and are
active readers and users of published code.

There are a good number of open source projects in economics producing excellent code already, 
as well as several projects aimed at teaching computational economics.
JOSEcon strives to unite these developers into the bigger scale community.

\begin{enumerate}
\item \href{https://econ-ark.org/}{Econ-ARK} by Christopher Carroll and the team;
\item \href{https://quantecon.org/}{QuantEcon} by Thomas Sargent, John Stachurski and the team;
\item \href{https://github.com/OpenSourceEconomics}{OpenSourceEconomics} by Philipp Eisenhauer and the team;
\item \href{https://github.com/janosg/skillmodels}{skillmodels} by Janos Gabler and the team;
\item \href{http://www.vfitoolkit.com/}{VFI toolkit} by Robert Kirkby;
\item \href{https://github.com/EconForge}{EconForge and Dolo library} by Pablo Winant;
\item \href{}{Hetsol toolkit} by Michael Rieter;
\item \href{https://www.dynare.org}{Dynare} by Michel Juillard and the team;
\item \href{http://www.macromodelbase.com}{Macroeconomic Model Data Base} by Volker Wielands;
\item \href{https://pypi.org/project/pyblp/}{pyblp package} by Christopher Conlon and Jeff Gortmaker;
\end{enumerate}

There are several ways that new members are recruited into the JOSEcon community.  
Existing faculty who are interested in computational methods would benefit from the lower 
barrier of entry thanks to JOSEcon collection of tested and reliable code.
In addition, the code published at JOSEcon would provide great resource in teaching.
Students working in the fields which have substantial computational component would
have a chance to start their careers by publishing in JOSEcon relatively quickly.
The younger students interested in the field could use their assignments and term papers
to replicate the results of published traditional papers, and also publish well
ahead of their job market.

Important other role models include:
\begin{itemize}
\item Referee for a different journal wishing to verify the computational part of 
the paper provided that is has been submitted to JOSEcon;
\item Member of editorial board of another journal to suggest independent code review 
in their editorial process;
\end{itemize}  


\subsection{Collaboration with other organizations}
%%%%%%%%%%%%%%%%%%%%%%%%%%%%%%%%%%%%%%%%%%%%%%%%%%%%%%%%%%%%%%%%%%%

JOSEcon is actively seeking collaborations with other organizations to popularize 
better practices in the use of numerical methods throughout the economic profession.

\paragraph{Societies:} 
Endorsements from various societies in economics help JOSEcon
gain respect in the profession.
\begin{itemize}
\item \href{https://comp-econ.org/}{Society of Computational Economics}
\end{itemize}


\paragraph{Existing software journals:} The JOSEcon editorial team carefully monitors
other software journals both to absorb better journal practices, and to monitor 
competition. 
\begin{enumerate}
\item \href{https://joss.theoj.org}{Journal of Open Source Software (JOSS)}
\begin{itemize}
\item Founded in May 2016
\item Developer-friendly, very short structured “papers”, focus of the code
\item Indexed in CrossRef, DOI identifiers
\item The founder and current editor-in-chief is \href{http://www.arfon.org/about/}{Arfon Smith}
\end{itemize}
\item \href{https://jose.theoj.org}{Journal of Open Source Education (JOSE)}
\item \href{https://www.jstatsoft.org}{Journal of Statistical Software (JSS)}
\item \href{http://rescience.github.io}{ReScience C journal}
\item \href{https://www.igi-global.com/journal/international-journal-open-source-software/1123}{International Journal of Open Source Software and Processes}
\item \href{https://ropensci.org}{rOpenSci}
\item \href{http://www.runmycode.org}{Run My Code community}
\item \href{https://www.journals.elsevier.com/software-impacts/}{Software Impacts}
\end{enumerate}

JOSEcon builds on the principles and infrastructure developed by 
\href{http://www.theoj.org/}{The Open Journals}, which includes JOSS and JOSE. 

\paragraph{Traditional journals:}
JOSEcon is actively seeking relationships with ``traditional'' journals, and hopes that
when it is reputable enough, such relationships would develop without outside force.

JOSEcon vision is that in the long term traditional journals would take into consideration
during their editorial process the publication at JOSEcon for the papers that rely heavily 
on computation.
Such collaboration would benefit the traditional journals, taking code review part off 
their shoulders. 

\paragraph{Publishing houses:} 
It is unclear how JOSEcon would coordinate with the existing publishing houses, being
absolutely open (both the publications, and the editorial process) to the public.
Yet, involvement of a respected brand would be invaluable for the development of 
the journal.

\paragraph{Open Source Initiative:}
JOSEcon is an affiliate of the \href{https://opensource.org/}{Open Source Initiative}.
\note{Fedor: formal affiliation pending}
As such JOSEcon is committed to public support for open source software and the role OSI plays therein.



