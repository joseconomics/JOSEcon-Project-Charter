%!TEX root = josecon_charter.tex

\subsection{Organizational structure and governance}
%%%%%%%%%%%%%%%%%%%%%%%%%%%%%%%%%%%%%%%%%%%%%%%%%%%%%%%%%%%%%%%%%%%

The governance structure at JOSEcon is primarily modeled from JOSS and Julia project,
and consists of an \emph{Editorial Board} running the every day operations at the 
journal, and an \emph{Advisory Board} overseeing these operations.

The editorial board consists of an \emph{Editor-in-Chief} and a team of \emph{Co-Editors}.
The members of the editorial board share the responsibilities of handling submitted papers, 
in particular they make the initial rejection decisions (individually or in private discussion
with the other editors, or by appointing external anonymous referees), 
and if the submission is not rejected right away, facilitate the open review process that
follows, and take the final decision.

The advisory board consists of prominent senior economists working in the broadly 
defined field of computational economics.  Currently the advisory board members are:
\begin{itemize}
\item Christopher Carroll
\item John Stachurski
\item John Rust
\item Felix K\"{u}bler
\item Serguei Maliar    
\end{itemize}


\subsection{Submission and review process}
%%%%%%%%%%%%%%%%%%%%%%%%%%%%%%%%%%%%%%%%%%%%%%%%%%%%%%%%%%%%%%%%%%%

JOSEcon operates similarly to the Journal of Open Source Software (JOSS), and together
with the core principles inherits the overall structure of the editorial process
which is described at \url{https://joss.readthedocs.io}

In particular, JOSEcon had clearly written guidelines for both authors, reviewers and editors.

The one difference between JOSEcon and JOSS is that the former has multiple 
submission types (above), and opts for a two stage review process:
\begin{enumerate}
\item Private review by editorial board (in some cases with anonymous referees) assessing 
the impact of the submission for the open source economics code;
\item Public review similar to JOSS focusing on code review.
\end{enumerate}

Once the submission passes through the first stage, it is typically never rejected.  Instead,
the reviewers help to ensure that the code satisfies the standards specified in the guidelines,
and even if the original authors take time to respond to the comments and suggestions, the 
code is eventually accepted to JOSEcon.

The accepted submissions are issued a JOSEcon badge that the authors are welcome 
to display at the code repository.
The code paper is published at JOSEcon website, and the accepted version of software is 
archived through the public archive services.

Accepted and published code can be developed further in the same repository, to the point
when the updates are sufficient to grant a new publication.

The web site of the journal is located at \url{http://josecon.org}. It contains the archive 
of all published papers with a topic index, and provides a full text search facility.

