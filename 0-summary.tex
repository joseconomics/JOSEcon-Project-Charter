%!TEX root = josecon_charter.tex

The Journal of Open Source Economics (JOSEcon) publishes ``code papers'' written by, 
and useful for the code developers in economics. 

The immediate goal of publishing coding work is to enable indexing and collection of citations by
its authors, thus allowing for the quality coding work to be credited similarly to traditional 
publications.
More boadly, JOSEcon promotes best software engineering practices in economics, 
building the computational economics community. 
It is also working together with traditional journals to help their review process 
of paper that rely heavily on computations.  

The field that has already built some mass of people aware and interested in better software development is computational macro, which has the majority of existing coding projects like QuantEcon or Econ-ARK.
But JOSEcon covers all fields of economics and econometrics that require computational methods,
such as microeconometrics and structural econometrics, empirical IO, life cycle modeling, Bayesian econometrics, computational game theory, etc. 

The journal is the economics field specific version of The Journal of Open Source Software (JOSS), 
and inherits its main principles: open publication of open source code with an open review process.
